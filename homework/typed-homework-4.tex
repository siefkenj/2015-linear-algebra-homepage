\documentclass[letter]{article}
\usepackage{amsmath}
\usepackage{amsfonts}
\usepackage{amssymb}
\usepackage{ifthen}
\usepackage{fancyhdr}
\usepackage[usenames,dvipsnames,svgnames,table]{xcolor}
\usepackage{tikz}

%%%
% Set up the margins to use a fairly large area of the page
%%%
\oddsidemargin=.2in
\evensidemargin=.2in
\textwidth=6in
\topmargin=0in
\textheight=9.0in
\parskip=.07in
\parindent=0in
\pagestyle{fancy}

\expandafter\def\expandafter\quote\expandafter{\quote\sf\color{DarkGreen}}

%%%
% Set up the header
%%%
\newcommand{\setheader}[6]{
	\lhead{{\sc #1}\\{\sc #2} %({\small \it \today})
	}
	\rhead{
		{\bf #3} 
		\ifthenelse{\equal{#4}{}}{}{(#4)}\\
		{\bf #5} 
		\ifthenelse{\equal{#6}{}}{}{(#6)}%
	}
}

%%%
% Set up some shortcut commands
%%%
\newcommand{\R}{\mathbb{R}}
\newcommand{\N}{\mathbb{N}}
\newcommand{\Z}{\mathbb{Z}}
\newcommand{\Proj}{\mathrm{proj}}
\newcommand{\Perp}{\mathrm{perp}}
\newcommand{\Span}{\mathrm{span}}
\newcommand{\Null}{\mathrm{null}}
\newcommand{\Rank}{\mathrm{rank}}
\newcommand{\mat}[1]{\begin{bmatrix}#1\end{bmatrix}}

%%%
% This is where the body of the document goes
%%%
\begin{document}
	\setheader{Math 211 (A01)}{Typed Homework 4}{Due Friday, March 27}{}{}{}

	\begin{enumerate}
		
		\item Look back on the worksheets and find a problem with at least three parts to it.
			Then, write up a solution to this problem.
			If the problem has more than three parts, you
			only need to explain your favorite three parts.

			Your answer must include
			a restatement of the problem, problem number, and all relevant definitions.  Further,
			include an introductory paragraph explaining why you chose this problem, and what learning
			objectives you think the problem deals with---at least one of the learning objectives you list
			must come from the list of learning objectives posted on the course webpage.
		
		\item We've come a long way in this course---we've gone from row-reduction to matrix equations, from
			spans and linear independence to eigenvectors and diagonalization.  I would like you to
			write about how this course fits together, what the main ideas are, and what the topics
			you found most difficult are.

			Imagine you are writing a letter to a parent/aunt/uncle who has 
			taken technical math/science courses in
			the past, but has forgotten most of the terminology.  Write to this audience explaining 
			what Math 211 is about, and highlight the main ideas in the class (when writing to 
			non-technical audience, examples are key).  In your essay, please do the following:
			\begin{enumerate}
				\item[(i)] address at least one
					of the three over-arching {\sc Learning Outcomes} listed on the course syllabus and 
					whether you feel you've achieved that learning outcome;
				\item[(ii)] include one technical math definition written in both math language and
					explained with plain English
				\item[(iii)] explain, for at least one linear algebra concept, how it can be viewed
					geometrically \emph{and} algebraically.
			\end{enumerate}

			Your essay should be 1--2 pages in length and start on a new page (i.e.,
			problem 1 gets its own page(s) and this essay gets its own pages).  To help you set the mood, 
			start your letter with,
			\begin{quote}
				Dear Mom, 

				This term at UVic I've been taking Math 211, which is a course about$\ldots$
			\end{quote}

			Of course, if you're inspired to start your essay a different way, feel free to!
			You should have fun with this essay.  After all, how often do you get a chance to explain 
			to others and yourself what a university course really taught you?

			Some guidelines on how your essay will be judged:
			\begin{enumerate}
				\item[Acceptable:]
					Correctly identified a learning outcome from the syllabus 
					and conveyed to the reader whether or not you have achieved it; gave a definition in mathematical
					and plain language; gave an example of a concept that we view geometrically and algebraically.
				\item[Good:]
					Correctly identified a learning outcome from the syllabus
					and explained clearly \emph{why} you have or have not achieved it; gave a definition in mathematical
					and plain language and explained \emph{how}
					this definition relates to solving linear algebra problems; 
					gave an example of a concept that we view geometrically and algebraically.
				\item[Excellent:]
					Correctly identified a learning outcome from the syllabus, 
					discussed \emph{how} this learning outcome connects to specific material or themes from the course, 
					and explained clearly \emph{why} you have or have not achieved it;  gave a definition in mathematical
					and plain language and explained \emph{how}
					this definition relates to solving linear algebra problems and \emph{how} definitions play an
					important role in linear algebra; gave an example of a concept that we view geometrically and algebraically
					and explained \emph{how} having a geometric and algebraic viewpoint of the same concept
					is useful.
			\end{enumerate}


	\end{enumerate}
\end{document}
