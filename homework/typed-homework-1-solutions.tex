\documentclass[letter]{article}
\usepackage{amsmath}
\usepackage{amsfonts}
\usepackage{amssymb}
\usepackage{ifthen}
\usepackage{fancyhdr}
\usepackage[usenames,dvipsnames,svgnames,table]{xcolor}
\usepackage{tikz}

%%%
% Set up the margins to use a fairly large area of the page
%%%
\oddsidemargin=.2in
\evensidemargin=.2in
\textwidth=6in
\topmargin=0in
\textheight=9.0in
\parskip=.07in
\parindent=0in
\pagestyle{fancy}

\expandafter\def\expandafter\quote\expandafter{\quote\sf\color{DarkGreen}}

%%%
% Set up the header
%%%
\newcommand{\setheader}[6]{
	\lhead{{\sc #1}\\{\sc #2} %({\small \it \today})
	}
	\rhead{
		{\bf #3} 
		\ifthenelse{\equal{#4}{}}{}{(#4)}\\
		{\bf #5} 
		\ifthenelse{\equal{#6}{}}{}{(#6)}%
	}
}

%%%
% Set up some shortcut commands
%%%
\newcommand{\R}{\mathbb{R}}
\newcommand{\N}{\mathbb{N}}
\newcommand{\Z}{\mathbb{Z}}
\newcommand{\Proj}{\mathrm{proj}}
\newcommand{\Perp}{\mathrm{perp}}
\newcommand{\Span}{\mathrm{span}}
\newcommand{\Null}{\mathrm{null}}
\newcommand{\Rank}{\mathrm{rank}}
\newcommand{\mat}[1]{\begin{bmatrix}#1\end{bmatrix}}

%%%
% This is where the body of the document goes
%%%
\begin{document}
	\setheader{Math 211 (A01)}{Typed Homework 1}{Due Friday, January 30}{}{}{}


	\begin{enumerate}
		\item Let $\vec u=\mat{1\\2\\3}$, $\vec v=\mat{4\\5\\6}$, and $\vec w=\mat{7\\8\\9}$. Explain whether
			the set $A=\{\vec u,\vec v,\vec w\}$ is a basis for $\R^3$.  
			Make sure to include all relevant definitions.

			\begin{quote}
				Recall a \emph{basis} for a subspace $V\subset \R^3$ is a linearly independent 
				set of vectors $B$ such that $\Span\, B=V$.  Let
				\[
					\vec u=\mat{1\\2\\3},\qquad 
					\vec v=\mat{4\\5\\6},\qquad \text{and}\qquad \vec w=\mat{7\\8\\9}.
				\]
				We will consider the question of whether $A=\{\vec u,\vec v,\vec w\}$ is a basis
				for $\R^3$.

				First, we will check to see if $A$ is a linearly independent set.  By definition,
				$A$ is \emph{linearly independent} if the only way to write
				$\vec 0$ as a linear combinations of vectors in $A$ is the trivial linear combination
				(the linear combination where all the coefficients are $0$).
				That is, we would like to know if
				\[
					\mat{0\\0\\0}=c_1\mat{1\\2\\3}+c_2\mat{4\\5\\6}+c_3\mat{7\\8\\9}
				\]
				has a solution other than $c_1=c_2=c_3=0$.  Rewriting this, we see
				\begin{align*}
					0&=c_1+4c_2+7c_3\\
					0&=2c_1+5c_2+8c_3\\
					0&=3c_1+6c_2+9c_3.
				\end{align*}
				Solving this system, we deduce that any $c_i$'s satisfying
				\[
					c_2=-2c_3=-2c_1
				\]
				give a solution.  Choosing $c_1=1$, we see that $(c_1,c_2,c_3)=(1,-2,1)$ is a solution.
				That is,
				\[
					\vec 0=\vec u-2\vec v+\vec w.
				\]
				Since $\vec 0$ could be written as a non-trivial linear combination 
				of vectors in $A$, $A$ is a linearly dependent
				set and therefore is not a basis for $\R^3$ (or any subspace).
			\end{quote}

		\item Fix $\vec u,\vec v\in \R^n$.  Show that $\Span(\Span\{\vec u,\vec v\})=\Span\{\vec u,\vec v\}$.
			Make sure to include all relevant definitions.
			\begin{quote}
				The \emph{span} of a set of vectors $X$ is the set of all linear combinations of 
				vectors in $X$.  We will show that for any two vectors $\vec u,\vec v\in \R^n$, 
				$\Span(\Span\{\vec u,\vec v\})=\Span\{\vec u,\vec v\}$.

				Fix $\vec u,\vec v\in \R^n$.  Let $S_1=\Span\{\vec u,\vec v\}$. By definition,
				\[
					S_1=\{\vec x\in\R^n:\vec x=\alpha\vec u+\beta\vec v\text{
					for some }\alpha,\beta\in \R\}.
				\]
				Let $\vec w\in \Span \,S_1$.  Again, by definition,
				\[
					\vec w=\sum_{i=1}^n \gamma_i\vec s_i
				\]
				for some $s_i\in S_1$ and $\gamma_i\in \R$.  Expanding based on the definition
				of $S_1$, we see
				\[
					\vec w =\sum_{i=1}^n \gamma_i (\alpha_i\vec u+\beta_i\vec v)=
					\left(\sum_{i=1}^n \gamma_i \alpha_i\right)\vec u+
					\left(\sum_{i=1}^n \gamma_i \beta_i\right)\vec v
				\]
				for some $\alpha_i,\beta_i\in \R$. However, the right hand sides expresses
				$\vec w$ as a linear combination of $\vec u$ and $\vec v$.  Thus, $\vec w\in S_1$.
				This shows $\Span\, S_1\subseteq S_1$.  Since $S_1 \subseteq \Span\,S_1$ is clear,
				we conclude $S_1=\Span\,S_1$.
			\end{quote}
		
		\item The worksheets define $\Proj_{\vec v}\vec u$ as the vector in the direction $\vec v$ such that
			$\vec u-\Proj_{\vec v}\vec u$ is orthogonal to $\vec v$.  Call this definition (a).  Your textbook
			defined $\Proj_{\vec v}\vec u$ as the vector $\frac{\vec u\cdot \vec v}{\vec v\cdot\vec v}\vec v$.
			Call this definition (b).  Show that definitions (a) and (b) are equivalent by showing 
			that that the vector arising from definition (b) must be the same as the vector
			arising from definition (a).
			In your answer, elaborate on definition (a) by including
			the definition of \emph{vector in the direction of $\vec v$} and \emph{orthogonal}.
			\begin{quote}
				Recall that $\vec u$ is \emph{in the same direction} as $\vec v\neq \vec 0$ if $\vec u=t\vec v$
				for some scalar $t$.  Further, $\vec u$ is \emph{orthogonal}
				to $\vec v$ if $\vec u\cdot \vec v=0$.

				Define
				\[
					\Proj_{1,\vec v}\vec u
				\]
				to be the vector in the direction $\vec v\neq\vec 0$ such that $\vec u-\Proj_{1,\vec v}\vec u$
				is orthogonal to $\vec v$.

				Define 
				\[
					\Proj_{2,\vec v}\vec u = \tfrac{\vec u\cdot \vec v}{\vec v\cdot \vec v}\vec v,
				\]
				provided $\vec v\neq \vec 0$.
				
				We aim to show that $\Proj_{2,\vec v}\vec u=\Proj_{1,\vec v}\vec u$.  Computing,
				\[
					\vec v\cdot(\vec u-\Proj_{2,\vec v}\vec u)=
					\vec v\cdot\left(\vec u-\tfrac{\vec u\cdot \vec v}{\vec v\cdot \vec v}\vec v\right)
					=\vec v\cdot\vec u - \tfrac{\vec u\cdot \vec v}{\vec v\cdot \vec v}(\vec v\cdot\vec v)
					=0,
				\]
				and so $\vec u-\Proj_{2,\vec v}\vec u$ is orthogonal to $\vec v$.
				Since $\tfrac{\vec u\cdot \vec v}{\vec v\cdot \vec v}$ is a scalar,
				$\Proj_{2,\vec v}\vec u=\tfrac{\vec u\cdot \vec v}{\vec v\cdot \vec v}\vec v$
				is a vector in the direction of $\vec v$.  Therefore, $\Proj_{2,\vec v}\vec u$
				satisfies the properties of $\Proj_{1,\vec v}\vec u$.  Since $\Proj_{1,\vec v}\vec u$
				is unique provided $\vec v\neq 0$, we conclude
				\[
					\Proj_{1,\vec v}\vec u = \Proj_{2,\vec v}\vec u.
				\]
			\end{quote}


	\end{enumerate}
\end{document}
