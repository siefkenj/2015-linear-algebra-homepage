\documentclass{article}
\usepackage{amsmath}
\usepackage{amsfonts}
\usepackage{amssymb}
\usepackage{ifthen}
\usepackage{fancyhdr}
\usepackage[usenames,dvipsnames,svgnames,table]{xcolor}
\usepackage{tikz}


\expandafter\def\expandafter\quote\expandafter{\quote\sf\color{DarkGreen}}


% In case you need to adjust margins:
\topmargin=-0.45in      %
\evensidemargin=0in     %
\oddsidemargin=0in      %
\textwidth=6.5in        %
\textheight=10.0in       %
\headsep=0.25in         %

\newcommand{\R}{\mathbb{R}}
\newcommand{\N}{\mathbb{N}}
\newcommand{\Z}{\mathbb{Z}}
\newcommand{\Proj}{\mathrm{proj}}
\newcommand{\Perp}{\mathrm{perp}}
\newcommand{\Span}{\mathrm{span}}
\newcommand{\Null}{\mathrm{null}}
\newcommand{\Rank}{\mathrm{rank}}
\newcommand{\mat}[1]{\begin{bmatrix}#1\end{bmatrix}}
\newenvironment{amatrix}[1]{%
  \left[\begin{array}{@{}*{#1}{c}|c@{}}
}{%
  \end{array}\right]
}


\begin{document}
	
	{\huge \bf Midterm 1} 
	
	\vspace{2em}
	{\huge MATH 211 (A01), Spring 2015 (Siefken)}
	
	\vspace{2em}
 	{
	\hfill \bf Date: \underline{\hspace{6em}}
	\vspace{.5em}

	\hfill Name: \underline{\hspace{16em}}
	\vspace{.5em}

	\hfill ID Number: \underline{\hspace{16em}}
	
	}

	\vspace{2in}
	This is a 50 minute test. It has 6 pages including this cover page.
	\begin{center}
	\begin{tabular}{|l|c|r|}
	\hline
	Q1 & $\phantom{abcd}$ & /10\\
	\hline
	Q2 & & /10\\
	\hline
	Q3 & & /10\\
	\hline
	Q4 & & /10\\
	\hline
	Q5 & & /10\\
	\hline
	\hline
	Total & & /50\\
	\hline
	\end{tabular}
	\end{center}
	\clearpage

\begin{enumerate}
	\item[1 (10pts)] Complete each of the following sentences with a mathematically precise definition.
	\begin{enumerate}
		\item[(a) (2pts)] A non-empty subset $V\subseteq \R^n$ is a {\bf subspace} if
			\begin{quote}
				for all $\vec v,\vec w\in V$ we have $\vec v+\vec w\in V$ and $k\vec v\in V$
				for all scalars $k$.
			\end{quote}
		\vspace{1in}
		\item[(b) (2pts)] The set of vectors $B=\{\vec v_1,\vec v_2,\vec v_3,\vec v_4\}$ is {\bf linearly independent} if
			\begin{quote}
				the only solution to 
				\[
					\alpha_1\vec v_1+\alpha_2\vec v_2+\alpha_3\vec v_3+\alpha_4\vec v_4=\vec 0
				\]
				is $\alpha_1=\alpha_2=\alpha_3=\alpha_4=0$.
			\end{quote}
		\vspace{.5in}
		\item[(c) (2pts)] The set of vectors $B=\{\vec v_1,\vec v_2,\vec v_3,\vec v_4\}$ is a {\bf basis for the subspace $V$} if
			\begin{quote}
				$B$ is linearly independent and $\Span\,B=V$.
			\end{quote}
		\vspace{1in}
		\item[(d) (2pts)] The vector $\vec a$ is a {\bf unit vector} if
			\begin{quote}
				$\|\vec a\|=1$.
			\end{quote}
		\vspace{1in}
		\item[(e) (2pts)] The vector $\vec a$ is {\bf orthogonal} to the vector $\vec b$ if
			\begin{quote}
				$\vec a\cdot \vec b=0$.
			\end{quote}
		\vspace{1in}
		
	
		%\item[(c) (6pts)] $\mathbf u$ and $\mathbf v$ are unit vectors and $\mathbf u\cdot \mathbf v = .765$.
		%Are $\mathbf u$ and $\mathbf v$ linearly indepenendent or linearly dependent.  {\bf Explain your answer.}
		%\item[(c) (6pts)] In 3d graphics, a triangle is defined by 3 points and a normal vector.  In order to save computational
		%time, a triangle is only displayed if the normal vector is pointing towards the viewer (and the triangle is in
		%front of the viewer).
		%
		%If you are positioned at the origin looking in the $[1\ 2\ 3]$ direction, should a triangle with
		%center $[30\ 60\ 20]$ and normal vector $[-2\ 2\ -1]$ be displayed?
		\vspace{5in}
	\end{enumerate}
	\clearpage

	\item[2 (10pts)] Let
		\[
			\vec a = \mat{1\\2\\3}\qquad \vec b=\mat{-1\\1\\1}\qquad \vec c=\mat{-3\\0\\-1}.
		\]
	\begin{enumerate}
		\item[(a) (3pts)] Compute $\vec a\cdot(\vec b-\vec c)$.
			\begin{quote}
				$\mat{1\\2\\3}\cdot \left(\mat{-1\\1\\1}-\mat{-3\\0\\-1}\right)=
				\mat{1\\2\\3}\cdot \mat{2\\1\\2}=10$.
			\end{quote}
		\vspace{.5in}
		\item[(b) (2pts)] Compute $-\vec a+2\vec b$.
			\begin{quote}
				$-\mat{1\\2\\3}+2\mat{-1\\1\\1}=\mat{-3\\0\\-1}$.
			\end{quote}
		\vspace{.5in}
		\item[(c) (5pts)] Is the set of vectors $X=\{\vec a,\vec b,\vec c\}$ linearly independent or linearly dependent?  Explain your
		answer.
			\begin{quote}
				$X$ is linearly dependent because $-\vec a+2\vec b-\vec c=\vec 0$ is a non-trivial linear combination of vectors
				in $X$ that equals $\vec 0$.
			\end{quote}
		\vspace{1in}
		
	
	\end{enumerate}
	\clearpage
	
	\item[3 (10pts)] $\phantom{xx}$
	\begin{enumerate}
		\item[(a) (7pts)]
		Use an augmented matrix to solve the following system of equations
		\[\begin{array}{rrrrrrc}
				&&   y &+&  z &   = & 1 \\
				x &   +&2y&+&  z &  = & 1 \\
				2x &  +& 4y&+&3 z  & = &  4 
		\end{array}.\]
		\begin{quote}
			\[
				\begin{amatrix}{3}
					0&1&1&1\\
					1&2&1&1\\
					2&4&3&4
				\end{amatrix}
				 \xrightarrow{R_1\leftrightarrow R_2}
				\begin{amatrix}{3}
					1&2&1&1\\
					0&1&1&1\\
					2&4&3&4
				\end{amatrix}
				 \xrightarrow{R_3\to R_3-2R_1}
				\begin{amatrix}{3}
					1&2&1&1\\
					0&1&1&1\\
					0&0&1&2
				\end{amatrix}
			\]
			\[
				 \xrightarrow{R_1\to R_1-2R_2}
				\begin{amatrix}{3}
					1&0&-1&-1\\
					0&1&1&1\\
					0&0&1&2
				\end{amatrix}
				 \xrightarrow{R_2\to R_2-2R_3}
				\begin{amatrix}{3}
					1&0&-1&-1\\
					0&1&0&-1\\
					0&0&1&2
				\end{amatrix}
				 \xrightarrow{R_1\to R_1+R_3}
				\begin{amatrix}{3}
					1&0&0&1\\
					0&1&0&-1\\
					0&0&1&2
				\end{amatrix}
			\]
			and so $(x,y,z)=(1,-1,2)$.
		\end{quote}

		\vspace{4in}

		\item[(b) (3pts)] Let $\vec a=\mat{0\\1\\2}$, $\vec b=\mat{1\\2\\4}$, $\vec c=\mat{1\\1\\3}$, and $\vec v=\mat{1\\1\\4}$.
			Express $\vec v$ as a linear combination of $\vec a$, $\vec b$, and $\vec c$. Hint, these vectors 
			closely relate to the system of equations in part (a).
			\begin{quote}
				$\vec v=\vec a-\vec b+2\vec c$.
			\end{quote}
	\end{enumerate}
	\clearpage

	\item[4 (10pts)] Below you are given the matrix $A$ and its reduced row echelon form.
		\[
			A=\mat{
				1 &  2 & -1 & -1 &  4\\
				2 &  4 &  1 &  0 &  4\\
				1 &  2 &  1 & -1 &  4
				   }\qquad
				\mathrm{rref}(A)=\mat{
				 1 &  2 &  0 &  0 &  2\\
				 0 &  0 &  1 &  0 &  0\\
				0 &  0 &  0 &  1 & -2}
		\]
		\item[(a) (2pts)]
			What is the rank of $A$?
			\begin{quote}
				$\Rank(A)=3$.
			\end{quote}

		\vspace{.5in}

	\item[(b) (2pts)] Write down the \emph{homogeneous system} of linear equations corresponding
			to $A$.
			\begin{quote}
				\[\begin{array}{rrrrrc}
						x_1&+2x_2&-x_3&-x_4&+4x_5&=0\\
						2x_1&+4x_2&+x_3& &+4x_5&=0\\
						x_1&+2x_2&+x_3&-x_4&+4x_5&=0\\
				\end{array}.\]

			\end{quote}

	\item[(c) (6pts)] Write down the \emph{general solution to} the homogeneous system of linear equations corresponding
			to $A$.
			\begin{quote}
				Since we've been given the row reduced form of $A$, we immediately indentify
				$x_2$ and $x_5$ as the only free variables.

				Let $x_2=t$ and $x_5=s$.  We then see
				\[
					\mat{x_1\\x_2\\x_3\\x_4\\x_5}=
					\mat{-2t-2s\\t\\0\\2s\\s}
					=
					t\mat{-2\\1\\0\\0\\0}+
					s\mat{-2\\0\\0\\2\\1}.
				\]
			\end{quote}


	\clearpage

	\item[5 (10pts)] For each of the following subsets of $\R^2$, either prove that the subset is a subspace
		or provide an example that shows it is not a subspace.
	\begin{enumerate}
		\item[(a) (5pts)] $U=\{\vec x\in \R^2: \vec x\cdot \vec e_1+\vec x\cdot \vec e_2 = 1\}$.
			\begin{quote}
				$U$ is not a subspace because it is not closed under vector addition.
				Notice $\vec e_1,\vec e_2\in U$, but $\vec w=\vec e_1+\vec e_2$ satisfies
				$\vec w\cdot \vec e_1+\vec w\cdot\vec e_2=2$ and so $\vec w\notin U$.
			\end{quote}
		\vspace{3.2in}
		\item[(b) (5pts)] $V=\{\vec x \in \R^2: \vec x\cdot \vec e_1+\vec x\cdot \vec e_2=0\}$.
			\begin{quote}
				$V$ is a subspace.  Let $\vec u,\vec v\in V$.  Then $\vec u\cdot \vec e_1+\vec u\cdot \vec e_2=0$
				and $\vec v\cdot \vec e_1+\vec v\cdot \vec e_2=0$.  We then have
				\[
					(\vec u+\vec v)\cdot \vec e_1+(\vec u+\vec v)\cdot \vec e_2=
					\vec u\cdot \vec e_1+\vec u\cdot \vec e_2 + \vec v\cdot \vec e_1+\vec v\cdot \vec e_2
					= 0+0 = 0,
				\]
				and so $V$ is closed under vector addition.  Further, if $\vec v\in V$, we have
				\[
					(k\vec v)\cdot \vec e_1+(k\vec v)\cdot \vec e_2=
					k(\vec v\cdot \vec e_1+\vec v\cdot \vec e_2)=k0=0
				\]
				for any scalar $k$, and so $V$ is closed under scalar multiplication.
			\end{quote}
	\end{enumerate}
	\clearpage


\end{enumerate}
\end{document}

