\documentclass[letter]{article}
\usepackage{amsmath}
\usepackage{amsfonts}
\usepackage{amssymb}
\usepackage{ifthen}
\usepackage{fancyhdr}
\usepackage[usenames,dvipsnames,svgnames,table]{xcolor}
\usepackage{tikz}

%%%
% Set up the margins to use a fairly large area of the page
%%%
\oddsidemargin=.2in
\evensidemargin=.2in
\textwidth=6in
\topmargin=0in
\textheight=9.0in
\parskip=.07in
\parindent=0in
\pagestyle{fancy}

\expandafter\def\expandafter\quote\expandafter{\quote\sf\color{DarkGreen}}

%%%
% Set up the header
%%%
\newcommand{\setheader}[6]{
	\lhead{{\sc #1}\\{\sc #2} %({\small \it \today})
	}
	\rhead{
		{\bf #3} 
		\ifthenelse{\equal{#4}{}}{}{(#4)}\\
		{\bf #5} 
		\ifthenelse{\equal{#6}{}}{}{(#6)}%
	}
}

%%%
% Set up some shortcut commands
%%%
\newcommand{\R}{\mathbb{R}}
\newcommand{\N}{\mathbb{N}}
\newcommand{\Z}{\mathbb{Z}}
\newcommand{\Proj}{\mathrm{proj}}
\newcommand{\Perp}{\mathrm{perp}}
\newcommand{\Span}{\mathrm{span}}
\newcommand{\Null}{\mathrm{null}}
\newcommand{\Rank}{\mathrm{rank}}
\newcommand{\mat}[1]{\begin{bmatrix}#1\end{bmatrix}}

%%%
% This is where the body of the document goes
%%%
\begin{document}
	\setheader{Math 211 (A01)}{Typed Homework 2}{Due Friday, February 20}{}{}{}

	\begin{enumerate}
		\item We know a system of linear equations can have $0,1$, or infinitely many solutions.
			\begin{enumerate}
				\item Explain why a system of linear equations cannot have exactly 2 solutions.
					\begin{quote}
						Let $\mathcal E$ be the system of linear equations given in vector
						form as
						\[
							\alpha_1\vec v_1+\alpha_2\vec v_2+\cdots+\alpha_n\vec v_n=\vec b,
						\]
						and suppose $\mathcal E$ has two solutions
						$\vec x=(x_1,x_2,\ldots,x_n)$ and $\vec y=(y_1,y_2,\ldots,y_n)$.
						Then, $\vec z=\frac{1}{2}(\vec x+\vec y)$ is another solution since
						\[
							\tfrac{1}{2}(x_1+y_1)\vec v_1+\tfrac{1}{2}(x_2+y_2)\vec v_2+\cdots+\tfrac{1}{2}(x_n+y_n)\vec v_n
						\]
						\[
							=
							\tfrac{1}{2}\Big(x_1\vec v_1+x_2\vec v_2+\cdots+x_n\vec v_n\Big)
							+\tfrac{1}{2}\Big(y_1\vec v_1+y_2\vec v_2+\cdots+y_n\vec v_n\Big)=\tfrac{1}{2}\vec b+\tfrac{1}{2}\vec b=\vec b.
						\]
						Thus if a system as two solutions it also has at least three solutions!

					\end{quote}
				\item For a homogeneous system of linear equations, what are the possibilities
					for the number of solutions?  Explain, and make sure to define 
					\emph{homogeneous system}.
					\begin{quote}
						Recall that a system of linear equations $\mathcal S$ is \emph{homogeneous} if
						every equation equals zero.  Let $\mathcal S$ be the homogeneous system of linear equations
						given in vector form by
						\[
							\alpha_1\vec v_1+\alpha_2\vec v_2+\cdots+\alpha_n\vec v_n=\vec 0.
						\]
						We see that this system always has at least one solution, namely the solution
						$\alpha_1=\alpha_2=\cdots=\alpha_n=0$.  If $\{\vec v_1,\vec v_2,\ldots,\vec v_n\}$
						is linearly dependent, $\mathcal S$ would have infinitely many solutions.  Thus,
						the number of possible solutions for a homogeneous system is $1$ or $\infty$, but never $0$.
					\end{quote}
			\end{enumerate}

		\item Suppose $\mathcal M$ is a homogeneous system of 4 linear equations and 3 variables.
			Let $M$ be the non-augmented matrix of coefficients.
			\begin{enumerate}
				\item If $\Rank(M)=3$, how many solutions does the system $\mathcal M$ have?
					(You do not need to define \emph{reduced row echelon form}, but include
					all other relevant definitions.)
					\begin{quote}
						Recall that a \emph{homogeneous} system of linear equations is one where
						every equation equals zero.  The \emph{rank} of a matrix is the number of leading
						ones in the reduced row echelon form of that matrix.

						A homogeneous system of linear equations has either $1$ or infinitely
						many solutions, and this directly corresponds to the number of 
						free variables in the system: if there are no free variables, the system has
						$1$ solution; if there are any free variables, the system has infinitely
						many solutions.  Since we choose free variables by picking columns
						of the reduced row echelon form of the coefficient matrix that do
						not have leading $1$'s and the matrix $M$ has $3$ columns and $3$ leading
						$1$'s (because the rank is $3$), we know there are no free variables for
						the system $\mathcal M$.  Thus, $\mathcal M$ has a unique solution.
					\end{quote}
				\item If $\Rank(M)=2$, how many solutions does the system $\mathcal M$ have?
					\begin{quote}
						As in part (a), the number of solutions is determined by the
						number of free variables.  Since $M$ has $3$ columns and $2$ leading
						$1$'s in its reduced row echelon form (since the rank of $M$ is 2), we have one free variable, and
						so $\mathcal M$ has infinitely many solutions.
					\end{quote}
				\item Could $\Rank(M)=4$? Explain.
					\begin{quote}
						In the reduced row echelon form of a matrix, every column can only have
						one leading $1$.  Since there are $3$ columns in $M$, $M$ has at most
						$3$ leading ones and so \[\Rank(M)\leq 3.\]
					\end{quote}
			\end{enumerate}
		
		\item Consider the equation $\mathcal E$ given by
			\[
				x_1-x_2+x_3-x_4=0
			\]
			and let $V=\left\{\mat{x_1\\x_2\\x_3\\x_4}\in \R^4:\mat{x_1\\x_2\\x_3\\x_4}\text{ is a solution to }\mathcal E\right\}$.
			Find vectors $\vec v_1,\ldots,\vec v_n$ such that $V=\Span\{\vec v_1,\ldots,\vec v_n\}$.  Explain your
			process.
			\begin{quote}
				Since $\mathcal E$ is a homogeneous equation, we may row-reduce a non-augmented
				matrix of coefficients to produce the set of solutions.  This gives us the matrix
				\[
					\mat{1&-1&1&-1}
				\]
				which has three free variable columns, namely $x_2$, $x_3$, $x_4$.  Let $x_2=t$, $x_3=s$, and $x_4=r$.
				We then see the solutions to this system are given by
				\[
					\mat{x_1\\x_2\\x_3\\x_4}=
					\mat{t-s+r\\t\\s\\r}=t\mat{1\\1\\0\\0}+s\mat{-1\\0\\1\\0}+r\mat{1\\0\\0\\1}
				\]
				for $t,r,s\in\R$.  However, the right hand side of the equation is exactly linear combinations
				of the vectors $\vec v_1=\mat{1\\1\\0\\0}$, $\vec v_2=\mat{-1\\0\\1\\0}$ and $\vec v_3=
				\mat{1\\0\\0\\1}$.  Thus,
				\[
					V=\Span\{\vec v_1,\vec v_2,\vec v_3\}.
				\]
			\end{quote}



	\end{enumerate}
\end{document}
