\documentclass[letter]{article}
\usepackage{amsmath}
\usepackage{amsfonts}
\usepackage{amssymb}
\usepackage{ifthen}
\usepackage{fancyhdr}
\usepackage[usenames,dvipsnames,svgnames,table]{xcolor}
\usepackage{tikz}

%%%
% Set up the margins to use a fairly large area of the page
%%%
\oddsidemargin=.2in
\evensidemargin=.2in
\textwidth=6in
\topmargin=0in
\textheight=9.0in
\parskip=.07in
\parindent=0in
\pagestyle{fancy}

\expandafter\def\expandafter\quote\expandafter{\quote\sf\color{DarkGreen}}

%%%
% Set up the header
%%%
\newcommand{\setheader}[6]{
	\lhead{{\sc #1}\\{\sc #2} %({\small \it \today})
	}
	\rhead{
		{\bf #3} 
		\ifthenelse{\equal{#4}{}}{}{(#4)}\\
		{\bf #5} 
		\ifthenelse{\equal{#6}{}}{}{(#6)}%
	}
}

%%%
% Set up some shortcut commands
%%%
\newcommand{\R}{\mathbb{R}}
\newcommand{\N}{\mathbb{N}}
\newcommand{\Z}{\mathbb{Z}}
\newcommand{\proj}{\mathrm{proj}}
\newcommand{\Span}{\mathrm{span}}
\newcommand{\Null}{\mathrm{null}}
\newcommand{\Rank}{\mathrm{rank}}
\newcommand{\Det}{\mathrm{det}}
\newcommand{\mat}[1]{\begin{bmatrix}#1\end{bmatrix}}

%%%
% This is where the body of the document goes
%%%
\begin{document}
	\setheader{Math 211 (A01)}{Written Homework 4}{Due Friday, March 20}{}{}{}

	\begin{enumerate}
		\item Suppose the matrix equation $A\vec x=\mat{3\\2\\7}$ has the general solution
			\[
				\vec x=\mat{1\\0\\0}+s\mat{1\\1\\0}+t\mat{-1\\0\\1}.
			\]
			\begin{enumerate}
				\item How many rows and how many columns does $A$ have?
				\item Find $\Null(A)$.
				\item Find $\Rank(A)$.
				\item Find $\text{col}(A)$.
				\item Find $\text{row}(A)$.
			\end{enumerate}
		
		\item Let
			\[
				\vec b_1=\mat{1\\1\\1}\qquad\vec b_2=\mat{1\\-1\\0}\qquad
				\vec b_3=\mat{-1\\0\\1}\qquad \vec c=\mat{1\\2\\3}
			\]
			and let $\mathcal B=\{\vec b_1,\vec b_2,\vec b_3\}$ and $\mathcal S=\{\vec e_1,\vec e_2,\vec e_3\}$.
			Suppose $T:\R^3\to\R^3$ is a linear transformation and $T(\vec b_1)=2\vec b_1$, 
			$T(\vec b_2)=3\vec b_2$, and $T(\vec b_3)=-\vec b_3$.
			\begin{enumerate}
				\item Compute $[\vec c]_{\mathcal B}$.
				\item Compute $[T\vec c]_{\mathcal B}$ and $[T\vec c]_{\mathcal S}$.
				\item Find a matrix for $T$ in the $\mathcal B$ basis (i.e., 
					the matrix $[T]_{\mathcal B}$) and a matrix for $T$ in
					the $\mathcal S$ basis (i.e., $[T]_{\mathcal S}$).
			\end{enumerate}
		
		\item Let $A=\mat{1&2&1\\1&1&1\\1&0&0}$ and $B=\mat{1&2&1\\1&1&1\\1&0&x}$.
		\begin{enumerate}
			\item Compute $\Det(A)$.
			\item Compute $\Det(B)$.  For what values of $x$ is $B$ not invertible?
		\end{enumerate}

		\item Let $A=\mat{1&2\\5&9}$.
		\begin{enumerate}
			\item Find an equation for the function $p(x)=\Det(A-xI)$ (this is called the
				\emph{characteristic polynomial} of $A$).
			\item For what values of $x$ is $A-xI$ non-invertible?
			\item Compute $p(A)$, the polynomial $p$ with the matrix $A$ plugged into it.  When you plug a matrix
				into a polynomial, replace any constant terms $k$ with the matrix $kI$.
				Can you guess
				why $p$ is called an \emph{annihilating} polynomial for $A$?
		\end{enumerate}

	\end{enumerate}
\end{document}
