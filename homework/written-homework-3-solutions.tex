\documentclass[letter]{article}
\usepackage{amsmath}
\usepackage{amsfonts}
\usepackage{amssymb}
\usepackage{ifthen}
\usepackage{fancyhdr}
\usepackage[usenames,dvipsnames,svgnames,table]{xcolor}
\usepackage{tikz}

%%%
% Set up the margins to use a fairly large area of the page
%%%
\oddsidemargin=.2in
\evensidemargin=.2in
\textwidth=6in
\topmargin=0in
\textheight=9.0in
\parskip=.07in
\parindent=0in
\pagestyle{fancy}

\expandafter\def\expandafter\quote\expandafter{\quote\sf\color{DarkGreen}}

%%%
% Set up the header
%%%
\newcommand{\setheader}[6]{
	\lhead{{\sc #1}\\{\sc #2} %({\small \it \today})
	}
	\rhead{
		{\bf #3} 
		\ifthenelse{\equal{#4}{}}{}{(#4)}\\
		{\bf #5} 
		\ifthenelse{\equal{#6}{}}{}{(#6)}%
	}
}

%%%
% Set up some shortcut commands
%%%
\newcommand{\R}{\mathbb{R}}
\newcommand{\N}{\mathbb{N}}
\newcommand{\Z}{\mathbb{Z}}
\newcommand{\proj}{\mathrm{proj}}
\newcommand{\Span}{\mathrm{span}}
\newcommand{\Null}{\mathrm{null}}
\newcommand{\Rank}{\mathrm{rank}}
\newcommand{\mat}[1]{\begin{bmatrix}#1\end{bmatrix}}

%%%
% This is where the body of the document goes
%%%
\begin{document}
	\setheader{Math 211 (A01)}{Written Homework 3}{Due Friday, February 27}{}{}{}

	\begin{enumerate}
		\item For each of the following statements, produce a counterexample to show that the statement is {\bf false}.
			\begin{enumerate}
				\item If $A$ and $B$ are square matrices, $AB=BA$.
					\begin{quote}
						\[
							\mat{1&1\\0&0}\mat{1&0\\1&0}=
							\mat{2&0\\0&0}\neq
							\mat{1&1\\1&1}=
							\mat{1&0\\1&0}\mat{1&1\\0&0}
						\]
					\end{quote}
				\item If $AB=\mat{1&1\\1&1}$, then $A$ and $B$ are $2\times 2$ matrices.
					\begin{quote}
						\[
							\mat{1\\1}\mat{1&1} = \mat{1&1\\1&1}
						\]
					\end{quote}
				\item If $AB=I$ then $BA=I$.
					\begin{quote}
						\[
							\mat{\frac{\sqrt{2}}{2}&\frac{\sqrt{2}}{2}}\mat{\frac{\sqrt{2}}{2}\\\frac{\sqrt{2}}{2}}
							=[1]=I_{1\times 1},
						\]
						but
						\[
							\mat{\frac{\sqrt{2}}{2}\\\frac{\sqrt{2}}{2}}\mat{\frac{\sqrt{2}}{2}&\frac{\sqrt{2}}{2}}
							=\frac{1}{2}\mat{1&1\\1&1}\neq I_{2\times 2}.
						\]
					\end{quote}
				\item If $A^2=0$, then $A=0$.
					\begin{quote}
						\[
							\mat{0&1\\0&0}^2=\mat{0&0\\0&0}=0_{2\times 2},
						\]
						but 
						\[
							\mat{0&1\\0&0}\neq 0_{2\times 2}.
						\]
					\end{quote}
			\end{enumerate}
		\item Let $R=\mat{1&2&3\\4&5&6\\7&8&9}$.
			\begin{enumerate}
				\item Find all solutions to the matrix equation $R\mat{x_1\\x_2\\x_3}=\mat{2\\5\\8}$.
					\begin{quote}
						The matrix equation $R\vec x=\mat{2\\5\\8}$ corresponds to the system
						of linear equations given by the augmented matrix
						\[
							A=\left[\begin{array}{ccc|c}
									1&2&3&2\\
									4&5&6&5\\
									7&8&9&8
								\end{array}\right].
						\]
						Row reducing, we find
						\[
							\mathrm{rref}(A) = 
							\left[\begin{array}{ccc|c}
									1&0&-1&0\\0&1&2&1\\0&0&0&0
								\end{array}\right].
						\]
						Identifying $x_3$ as a free variable and adding the equation $x_3=t$, we
						produce a general solution of
						\[
							\mat{x_1\\x_2\\x_3} = t\mat{1\\-2\\1}+\mat{0\\1\\0}
						\]
						for $t\in \R$.
					\end{quote}
				\item Prove that the set $X=\{\vec x\in\R^3:R\vec x=\vec 0\}$ is a subspace.
					\begin{quote}
						Suppose $\vec u,\vec v\in X$.  That means $R\vec u=R\vec v=\vec 0$.  
						Considering $\vec u+\vec v$ we see
						\[
							R(\vec u+\vec v) = R\vec u+R\vec v=\vec 0+\vec 0=\vec 0,
						\]
						and so $\vec u+\vec v\in X$.
						Further, if $k\in \R$, then 
						\[
							R(k\vec u) = kR\vec u= k\vec 0=\vec 0,
						\]
						and so $k\vec u\in X$, showing that $X$ is a subspace.
					\end{quote}
			\end{enumerate}
		\item Suppose $E$ is a $4\times 3$ matrix with columns $\vec c_1,\vec c_2,\vec c_3$ and rows
			$\vec r_1,\vec r_2,\vec r_3,\vec r_4$.  Let $\vec v=\mat{2\\-1\\1}$.
			\begin{enumerate}
				\item Express $E\vec v$ as a linear combination of $\vec c_1,\vec c_2,\vec c_3$.
					\begin{quote}
						\[
							E\vec v = [\vec c_1|\vec c_2|\vec c_3]\mat{2\\-1\\1} = 2\vec c_1-\vec c_2+\vec c_3.
						\]
					\end{quote}
				\item Supposing $\vec r_1\cdot \vec v=1$, $\vec r_2\cdot \vec v=6$, $(\vec r_3+\vec r_4)\cdot \vec v=2$,
					and $(\vec r_3-\vec r_4)\cdot \vec v=-2$, compute $E\vec v$.
					\begin{quote}
						Considering matrix multiplication as dot products of rows of the first matrix with columns
						of the second, we see
						\[
							E\vec v = \mat{ \vec r_1\cdot\vec v\\
									\vec r_2\cdot\vec v\\
									\vec r_3\cdot\vec v\\
									\vec r_4\cdot\vec v}.
						\]
						The only values we haven't been given are
						$\vec r_3\cdot \vec v$ and $\vec r_4\cdot \vec v$, but by using the equations
						$(\vec r_3+\vec r_4)\cdot \vec v=2$
						and $(\vec r_3-\vec r_4)\cdot \vec v=-2$, we deduce that
						\[
							\vec r_3\cdot \vec v = 0\qquad\text{and}\qquad
							\vec r_4\cdot \vec v =2.
						\]
						Thus,
						\[
							E\vec v = \mat{1\\6\\0\\2}.
						\]


					\end{quote}
			\end{enumerate}

		\item Suppose that $\vec u$, $\vec v$, and $\vec w$ are vectors in $\R^2$ that are related
			by the following diagram.
		\begin{center}
			\begin{tikzpicture}[scale=.5]
				\draw[-] (-6, 0) -- node [below,
				very near end] {} (6, 0);
				\draw[-] (0, -2) -- node [right,
				very near start] {} (0, 5);
				\draw[->] (0, 0) -- node [below,
				very near end] {$4\vec{u}$}
				(3, -1);
				\draw[->] (0, 0) -- node [left,
				near end] {$2\vec{v}$} (2, 3);
				\draw[->] (0, 0) -- node [left,
				very near end] {$3\vec{w}$}
				(-1, 4);
				\draw[dotted] (3, -1) -- node
				[right, very near end] {} (2, 3);
				\draw[dotted] (-1,4) -- node
				[below, very near end] {} (2, 3);
			\end{tikzpicture}
		\end{center}
		Let $A=[\vec u|\vec v|\vec w]$ be the matrix with columns $\vec u$, $\vec v$, and $\vec w$.
		\begin{enumerate}
			\item What is the rank of $A$?
					\begin{quote}
						Notice that $\{\vec u,\vec v\}$ are linearly independent,
						but $4\vec u-2\vec v+3\vec w=\vec 0$ and so $\{\vec u,\vec v,\vec w\}$
						is linearly dependent.  Thus, there are two linearly independent columns
						in $A$ and so $\Rank(A)=2$.
					\end{quote}
			\item Find all solutions to the equation $A\vec x=\vec 0$.
					\begin{quote}
						Since $4\vec u-2\vec v+3\vec w=\vec 0$, we know that
						\[
							A\mat{4\\-2\\3}=[\vec u|\vec v|\vec w]\mat{4\\-2\\3}=
							4\vec u-2\vec v+3\vec w=\vec 0.
						\]
						In fact, any multiple of $\mat{4\\-2\\3}$ is also a solution to
						$A\vec x=\vec 0$.  Since $\mathrm{nullity}(A)+\Rank(A)=\#$of columns of $A$,
						we know $\mathrm{nullity}(A)=1$.  Thus, all solutions to $A\vec x=\vec 0$ are given by
						\[
							x=t\mat{4\\-2\\3}
						\]
						for some $t\in \R$.
					\end{quote}
			\item Find a basis for the subspace $V=\{\vec x\in\R^3: A\vec x=\vec 0\}$.
					\begin{quote}
						As noted earlier, $V$ consists precisely of multiples of $\mat{4\\-2\\3}$ and
						so a basis is
						$
							\left\{\mat{4\\-2\\3}\right\}.
						$
					\end{quote}
		\end{enumerate}

	\end{enumerate}
\end{document}
