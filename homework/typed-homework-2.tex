\documentclass[letter]{article}
\usepackage{amsmath}
\usepackage{amsfonts}
\usepackage{amssymb}
\usepackage{ifthen}
\usepackage{fancyhdr}
\usepackage[usenames,dvipsnames,svgnames,table]{xcolor}
\usepackage{tikz}

%%%
% Set up the margins to use a fairly large area of the page
%%%
\oddsidemargin=.2in
\evensidemargin=.2in
\textwidth=6in
\topmargin=0in
\textheight=9.0in
\parskip=.07in
\parindent=0in
\pagestyle{fancy}

\expandafter\def\expandafter\quote\expandafter{\quote\sf\color{DarkGreen}}

%%%
% Set up the header
%%%
\newcommand{\setheader}[6]{
	\lhead{{\sc #1}\\{\sc #2} %({\small \it \today})
	}
	\rhead{
		{\bf #3} 
		\ifthenelse{\equal{#4}{}}{}{(#4)}\\
		{\bf #5} 
		\ifthenelse{\equal{#6}{}}{}{(#6)}%
	}
}

%%%
% Set up some shortcut commands
%%%
\newcommand{\R}{\mathbb{R}}
\newcommand{\N}{\mathbb{N}}
\newcommand{\Z}{\mathbb{Z}}
\newcommand{\Proj}{\mathrm{proj}}
\newcommand{\Perp}{\mathrm{perp}}
\newcommand{\Span}{\mathrm{span}}
\newcommand{\Null}{\mathrm{null}}
\newcommand{\Rank}{\mathrm{rank}}
\newcommand{\mat}[1]{\begin{bmatrix}#1\end{bmatrix}}

%%%
% This is where the body of the document goes
%%%
\begin{document}
	\setheader{Math 211 (A01)}{Typed Homework 2}{Due Friday, February 20}{}{}{}

	\begin{enumerate}
		\item We know a system of linear equations can have $0,1$, or infinitely many solutions.
			\begin{enumerate}
				\item Explain why a system of linear equations cannot have exactly 2 solutions.
				\item For a homogeneous system of linear equations, what are the possibilities
					for the number of solutions?  Explain, and make sure to define 
					\emph{homogeneous system}.
			\end{enumerate}

		\item Suppose $\mathcal M$ is a homogeneous system of 4 linear equations and 3 variables.
			Let $M$ be the non-augmented matrix of coefficients.
			\begin{enumerate}
				\item If $\Rank(M)=3$, how many solutions does the system $\mathcal M$ have?
					(You do not need to define \emph{reduced row echelon form}, but include
					all other relevant definitions.)
				\item If $\Rank(M)=2$, how many solutions does the system $\mathcal M$ have?
				\item Could $\Rank(M)=4$? Explain.
			\end{enumerate}
		
		\item Consider the equation $\mathcal E$ given by
			\[
				x_1-x_2+x_3-x_4=0
			\]
			and let $V=\left\{\mat{x_1\\x_2\\x_3\\x_4}\in \R^4:\mat{x_1\\x_2\\x_3\\x_4}\text{ is a solution to }\mathcal E\right\}$.
			Find vectors $\vec v_1,\ldots,\vec v_n$ such that $V=\Span\{\vec v_1,\ldots,\vec v_n\}$.  Explain your
			process.



	\end{enumerate}
\end{document}
