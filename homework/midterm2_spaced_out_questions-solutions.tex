\documentclass{article}
\usepackage{amsmath}
\usepackage{amsfonts}
\usepackage{amssymb}
\usepackage{ifthen}
\usepackage{fancyhdr}
\usepackage{enumerate}
\usepackage[usenames,dvipsnames,svgnames,table]{xcolor}
\usepackage{tikz}


\expandafter\def\expandafter\quote\expandafter{\quote\sf\color{DarkGreen}}

% In case you need to adjust margins:
\topmargin=-0.45in      %
\evensidemargin=0in     %
\oddsidemargin=0in      %
\textwidth=6.5in        %
\textheight=10.0in       %
	\headsep=0.25in         %

\newcommand{\R}{\mathbb{R}}
\newcommand{\N}{\mathbb{N}}
\newcommand{\Z}{\mathbb{Z}}
\newcommand{\Proj}{\mathrm{proj}}
\newcommand{\Perp}{\mathrm{perp}}
\newcommand{\Span}{\mathrm{span}}
\newcommand{\Null}{\mathrm{null}}
\newcommand{\Rank}{\mathrm{rank}}
\newcommand{\mat}[1]{\begin{bmatrix}#1\end{bmatrix}}
\newenvironment{amatrix}[1]{%
  \left[\begin{array}{@{}*{#1}{c}|c@{}}
}{%
  \end{array}\right]
}

\begin{document}
	
	{\huge \bf Midterm 2} 
	
	\vspace{2em}
	{\huge MATH 211 (A01), Spring 2015 (Siefken)}
	
	\vspace{2em}
 	{
	\hfill \bf Date: \underline{\hspace{6em}}
	\vspace{.5em}

	\hfill Name: \underline{\hspace{16em}}
	\vspace{.5em}

	\hfill ID Number: \underline{\hspace{16em}}
	
	}

	\vspace{2in}
	This is a 50 minute test. It has 6 pages including this cover page.
	\begin{center}
	\begin{tabular}{|l|c|r|}
	\hline
	Q1 & $\phantom{abcd}$ & /10\\
	\hline
	Q2 & & /10\\
	\hline
	Q3 & & /10\\
	\hline
	Q4 & & /10\\
	\hline
	Q5 & & /10\\
	\hline
	\hline
	Total & & /50\\
	\hline
	\end{tabular}
	\end{center}
	\clearpage

\begin{enumerate}
	\item[1 (10pts)] Complete each of the following sentences with a mathematically precise definition.
	\begin{enumerate}
		\item[(a) (2pts)] A non-empty subset $V\subseteq \R^n$ is a {\bf subspace} if
		\begin{quote}
			\begin{enumerate}[(i)]
				\item $\vec u,\vec v\in V$ implies $\vec u+\vec v\in V$ and
				\item $\vec u\in V$ implies $k\vec u\in V$ for any scalar $k$.
			\end{enumerate}
		\end{quote}
		\item[(b) (2pts)] A transfomration $T:\R^n\to\R^m$ is a {\bf linear transformation} if
		\begin{quote}
			\begin{enumerate}[(i)]
				\item $T(\vec u+\vec v)=T\vec u+T\vec v$ and
				\item $T(k\vec u)=kT(\vec u)$ for any scalar $k$.
			\end{enumerate}
		\end{quote}
		\item[(c) (2pts)] The {\bf null space} of a matrix $M$ is
		\begin{quote}
			\[
				\{\vec x:M\vec x=\vec 0\}.
			\]
		\end{quote}
		\item[(d) (2pts)] The {\bf inverse} of a matrix $A$ is \\ {\it(Hint: you will get no points if all you write is $A^{-1}$)}

		\begin{quote}
			a matrix $B$ such that $AB=BA=I$.
		\end{quote}
		\item[(e) (2pts)] The {\bf range} (or {\bf image}) of a linear transformation $T:\R^n\to\R^m$ is
		\begin{quote}
			\[
				\{\vec x\in \R^m:\vec x=T(\vec y)\text{ for some }\vec y\in \R^n\}.
			\]
		\end{quote}
		
	
		%\item[(c) (6pts)] $\mathbf u$ and $\mathbf v$ are unit vectors and $\mathbf u\cdot \mathbf v = .765$.
		%Are $\mathbf u$ and $\mathbf v$ linearly indepenendent or linearly dependent.  {\bf Explain your answer.}
		%\item[(c) (6pts)] In 3d graphics, a triangle is defined by 3 points and a normal vector.  In order to save computational
		%time, a triangle is only displayed if the normal vector is pointing towards the viewer (and the triangle is in
		%front of the viewer).
		%
		%If you are positioned at the origin looking in the $[1\ 2\ 3]$ direction, should a triangle with
		%center $[30\ 60\ 20]$ and normal vector $[-2\ 2\ -1]$ be displayed?
		\vspace{5in}
	\end{enumerate}
	\clearpage

	\item[2 (10pts)] Given
		\[
			A=\mat{1&1\\2&3}\qquad
			B=\mat{-1&1\\1&-1}\qquad
			C=\mat{1&2&-1\\2&2&4\\1&3&-3}\qquad
			C^{-1}=\mat{9&-3/2&-5\\-5&1&3\\-2&1/2&1}
		\]
	\begin{enumerate}
		\item[(a) (2pts)] Compute $AB$.
		\begin{quote}
			$AB=\mat{1&1\\2&3}\mat{-1&1\\1&-1}=\mat{0&0\\1&-1}$.
		\end{quote}
			\vspace{1.5in}
		\item[(b) (2pts)] Compute $A^{-1}$.
		\begin{quote}
			$A^{-1}=\mat{3&-1\\-2&1}$.
		\end{quote}
			\vspace{1.5in}
		\item[(c) (2pts)] Compute $C^T$.
		\begin{quote}
			$C^T=\mat{1&2&1\\2&2&3\\-1&4&-3}$.
		\end{quote}
			\vspace{1.5in}
		\item[(d) (4pts)] Solve the equation $C\vec x=\mat{1\\0\\1}$.
		\begin{quote}
			$C\vec x=\mat{1\\0\\1}$ is solved by 
			\[
			\vec x=C^{-1}
			\mat{1\\0\\1}=
			\mat{9&-3/2&-5\\-5&1&3\\-2&1/2&1}
			\mat{1\\0\\1} = \mat{4\\-2\\-1}.
			\]
		\end{quote}
		
	
	\end{enumerate}
	\clearpage
	
\item[3 (10pts)] For each of the following, indicate {\bf true} or {\bf false}.  You {\it do not}
	need to explain how you arrived at your answer.
	\begin{enumerate}
		%\item[(a) (3pts)] Let $M$ be an $n\times n$ matrix.  If $M\neq 0$, then we could have
		%	\begin{enumerate}[(i)]
		%		\item $\mathrm{row}(M)=\{\vec 0\}$
		%		\vspace{.65in}
		%		\item $\mathrm{col}(M)=\{\vec 0\}$
		%		\vspace{.65in}
		%		\item $\mathrm{null}(M)=\{\vec 0\}$ 
		%		\vspace{.65in}
		%	\end{enumerate}
		%\item[(b) (7pts)] Let $M$ be an $n\times n$ matrix.  If $M$ is invertible, then we could have
		\item[] Let $M$ be an $n\times n$ matrix.  If $M$ is invertible, then we \emph{must} have
			\begin{enumerate}[(i)]
				\item $\mathrm{row}(M)=\{\vec 0\}$
				\begin{quote}
					False
				\end{quote}
				\vspace{.65in}
				\item $\mathrm{col}(M)=\R^n$
				\begin{quote}
					True
				\end{quote}
				\vspace{.65in}
				\item $\mathrm{null}(M)=\{\vec 0\}$ 
				\begin{quote}
					True
				\end{quote}
				\vspace{.65in}
				%\item $M=E_1E_2E_3$ for elementary matrices $E_1$, $E_2$, $E_3$.
				%\vspace{.65in}
				%\item $\mathrm{rref}(M)=\mat{1&0&0\\0&1&0\\0&0&0}$
				%\vspace{.65in}
				\item $\mathrm{rank}(M)=n$
				\begin{quote}
					True
				\end{quote}
				\vspace{.65in}
				\item $M=M^T$ 
				\begin{quote}
					False
				\end{quote}
			\end{enumerate}
	\end{enumerate}
	\pagebreak

	\item[4 (10pts)] Let $\mathcal F:\R^2\to\R^2$ be the linear transformation that reflects
		vectors across the line with direction vector $\vec e_2$.  
		%Let $\mathcal R:\R^2\to\R^2$
		%be a counter-clockwise rotation by $45^\circ$.  
		Let $\mathcal P:\R^2\to\R^2$ be projection
		onto the vector $\mat{1\\2}$.
	\begin{enumerate}
		\item[(a) (4pt)] Compute $\mathcal F\mat{7\\-3}$ and $\mathcal P\mat{1\\0}$.
			\begin{quote}
				$\mathcal F\mat{7\\-3}=\mat{-7\\-3}$ and $\mathcal P\mat{1\\0}=
				\frac{\mat{1\\2}\cdot\mat{1\\0}}{\mat{1\\2}\cdot\mat{1\\2}}\mat{1\\2}=\frac{1}{5}\mat{1\\2}$.
			\end{quote}

		\vspace{1in}

		\item[(b) (6pts)] Find a matrix for $\mathcal F\circ \mathcal P$.
			\begin{quote}
				Computing $\mathcal F\mat{1\\0}=\mat{-1\\0}$ and $\mathcal F\mat{0\\1}=\mat{0\\1}$,
				we see that a matrix for $\mathcal F$ is 
				\[
					F=\mat{-1&0\\0&1}.
				\]
				Computing $\mathcal P\mat{1\\0}=\frac{1}{5}\mat{1\\2}$ and $\mathcal P\mat{0\\1}=\frac{2}{5}\mat{1\\2}$,
				we see that a matrix for $\mathcal P$ is 
				\[
					P=\frac{1}{5}\mat{1&2\\2&4}.
				\]
				Since the matrix representing a coposition of linear functions is the same as the product
				of the matrices representing that linear transformation, we see a matrix for $\mathcal F\circ \mathcal P$
				is
				\[
					FP=\frac{1}{5}\mat{-1&-2\\2&4}.
				\]
			\end{quote}
			\vspace{5in}
		%\item[(c) (2pts)] Is $\mathcal F\circ \mathcal P$ invertible?  Explain.
		%	\vspace{1in}
		%\item[(d) (2pts)] Is $\mathcal F\circ \mathcal R$ invertible?  Explain.
		%	\vspace{1in}
	\end{enumerate}
	\clearpage

	\item[5 (10pts)] For each of the following transformations, either prove that the transformation is linear
		or provide an example that shows it is not linear.
	\begin{enumerate}
		\item[(a) (5pts)] $U:\R^2\to\R^2$ defined by
			$U\mat{x\\y}=\mat{x\\x+y}$.
			\begin{quote}
				We will verify that $U$ distributes with respect to vector addition and scalar multiplication.
				Pick $\mat{x_1\\x_2},\mat{y_1\\y_2}\in\R^2$.  Then
				\[
					U\left(\mat{x_1\\x_2}+\mat{y_1\\y_2}\right)=U\mat{x_1+y_1\\x_2+y_2}=
					\mat{x_1+y_1\\x_1+y_1+x_2+y_2}
				\]\[
					=\mat{x_1\\x_1+x_2}+\mat{y_1\\y_1+y_2}=U\mat{x_1\\x_2}+U\mat{y_1\\y_2}.
				\]
				Similarly,
				\[
					U\left(k\mat{x_1\\x_2}\right) = U\mat{kx_1\\kx_2}=\mat{kx_1\\kx_1+kx_2} = 
					k\mat{x_1\\x_1+x_2} = kU\mat{x_1\\x_2},
				\]
				and so $U$ is linear.
			\end{quote}
		\vspace{1in}
		\item[(b) (5pts)] $V:\R^2\to\R^2$ defined by 
			$V\mat{x\\y}=\mat{x\\x-1}$.
			\begin{quote}
				$V$ is not linear because $V(\vec 0+\vec 0) =V(\vec 0)= \mat{0\\-1}$, but $V(\vec 0)+V(\vec 0) = \mat{0\\-2}$,
				and so $V$ does not distribute with respect to vector addition.
			\end{quote}
	\end{enumerate}
	\clearpage


\end{enumerate}
\end{document}

