\documentclass[letter]{article}
\usepackage{amsmath}
\usepackage{amsfonts}
\usepackage{amssymb}
\usepackage{ifthen}
\usepackage{fancyhdr}
\usepackage[usenames,dvipsnames,svgnames,table]{xcolor}
\usepackage{tikz}

%%%
% Set up the margins to use a fairly large area of the page
%%%
\oddsidemargin=.2in
\evensidemargin=.2in
\textwidth=6in
\topmargin=0in
\textheight=9.0in
\parskip=.07in
\parindent=0in
\pagestyle{fancy}

\expandafter\def\expandafter\quote\expandafter{\quote\sf\color{DarkGreen}}

%%%
% Set up the header
%%%
\newcommand{\setheader}[6]{
	\lhead{{\sc #1}\\{\sc #2} %({\small \it \today})
	}
	\rhead{
		{\bf #3} 
		\ifthenelse{\equal{#4}{}}{}{(#4)}\\
		{\bf #5} 
		\ifthenelse{\equal{#6}{}}{}{(#6)}%
	}
}

%%%
% Set up some shortcut commands
%%%
\newcommand{\R}{\mathbb{R}}
\newcommand{\N}{\mathbb{N}}
\newcommand{\Z}{\mathbb{Z}}
\newcommand{\Proj}{\mathrm{proj}}
\newcommand{\Perp}{\mathrm{perp}}
\newcommand{\Span}{\mathrm{span}}
\newcommand{\Null}{\mathrm{null}}
\newcommand{\Nullity}{\mathrm{nullity}}
\newcommand{\Rank}{\mathrm{rank}}
\newcommand{\Rref}{\mathrm{rref}}
\newcommand{\Range}{\mathrm{range}}
\newcommand{\mat}[1]{\begin{bmatrix}#1\end{bmatrix}}

%%%
% This is where the body of the document goes
%%%
\begin{document}
	\setheader{Math 211 (A01)}{Typed Homework 3}{Due Friday, March 6}{}{}{}

	\begin{enumerate}
		\item Let $T:\R^n\to\R^m$ be a linear transformation.
		\begin{enumerate}
			\item Show that the null space of $T$ is a subspace of $\R^n$.
				\begin{quote}
					Let $T:\R^n\to\R^m$ be a linear transformation.  By definition,
					$T$ is \emph{linear} if 
					\[
						T(\vec u+\vec v)=T\vec u+T\vec v\qquad\text{and}\qquad
						T(k\vec u)=kT\vec u
					\]
					for any $\vec u,\vec v\in \R^n$ and any scalar $k$.
					By definition, the \emph{null space} of a linear transformation
					$T:\R^n\to\R^m$ is the set $\Null(T)=\{\vec x\in\R^n:T\vec x=\vec 0\}$.
					We will show $\Null(T)$ satisfies the two conditions of a subspace,
					namely that it is closed under vector addition and scalar multiplication.

					Pick $\vec u,\vec v\in \Null(T)$.  That means, $T\vec u=T\vec v=\vec 0$.
					Now 
					\[
						T(\vec u+\vec v)=T\vec u+T\vec v=\vec 0+\vec 0=\vec 0,
					\] with the first
					equality following from the linearity of $T$.  We conclude $\vec u+\vec v\in\Null(T)$.

					Further, we see that
					\[
						T(k\vec u)=kT\vec u=k\vec 0=\vec 0,
					\]
					again with the first equality following from linearity.  From this we may
					conclude that $k\vec u\in \Null(T)$ for any scalar $k$, and so $\Null(T)$
					is a subspace.
				\end{quote}
			\item Show that the range of $T$ is a subspace of $\R^m$.
				\begin{quote}
					By definition, the \emph{range} of a linear transformation
					$T:\R^n\to \R^m$ is the set $\Range(T)=\{\vec x\in\R^m: \vec x=T\vec y\text{ for some }\vec y
					\in \R^n\}$.

					Proceeding in the standard way, pick $\vec x,\vec y\in \Range(T)$.  By definition,
					this means there are vectors $\vec x',\vec y'\in \R^n$ such that $\vec x=T(\vec x')$ and
					$\vec y=T(\vec y')$.

					By linearity of $T$, we have
					\[
						T(\vec x'+\vec y') = T(\vec x')+T(\vec y') = \vec x+\vec y.
					\]
					Therefore $\vec x+\vec y\in \Range(T)$ since we have found a vector
					$\vec w'=\vec x'+\vec y'$ so that $\vec x+\vec y=T(\vec w')$.

					Similarly, linearity of $T$ gives us that
					\[
						T(k\vec x')=kT(\vec x')=k\vec x,
					\]
					and so $k\vec x\in \Range(T)$ because the vectors $\vec w'=k\vec x'$ satisfies
					$k\vec x=T(\vec w')$.

				\end{quote}
		\end{enumerate}
		
		\item 
		\begin{enumerate}
			\item For a $4\times 3$ matrix $M$, must the column space of $M$ be identical to
				the column space of $\mathrm{rref}(M)$?
				\begin{quote}
					For a $4\times 3$ matrix, the column space of $M$ and $\Rref(M)$ may be different.
					By definition, the column space of a matrix is the span of the columns.  Consider the matrix
					\[
						M=\mat{0&0&0\\1&0&0\\0&0&0\\0&0&0}\qquad
						\text{and}
						\quad
						\Rref(M)=\mat{1&0&0\\0&0&0\\0&0&0\\0&0&0}.
					\]
					Here we have the column space of $M$ is the span of $\mat{0\\1\\0\\0}$
					but the column space of $\Rref(M)$ is the span of $\mat{1\\0\\0\\0}$,
					which is a completely different subspace!
				\end{quote}
			\item For a $3\times 3$ matrix $N$ with $\Rank(N)=3$, must the column space of $N$
				be identical to the column space of $\mathrm{rref}(N)$?  Can the assumption
				that $\Rank(N)=3$ be dropped?
				\begin{quote}
					The \emph{rank} of a matrix $N$ is the number of ones in its reduced
					row echelon form.  Equivalently, it is the number of linearly
					independent columns of $N$.

					Let $\vec c_1,\vec c_2,\vec c_3$ be the columns of $N$.  Since
					$N$ has rank 3, $\mathcal C=\{\vec c_1,\vec c_2,\vec c_3\}$ is a linearly independent
					set.  Since $N$ is a $3\times 3$ matrix, $\vec c_1,\vec c_2,\vec c_3\in \R^3$.
					Therefore $\mathcal C$ is a basis for $\R^3$ and so the column space of $N$ is all of
					$\R^3$.

					Alternatively, since the rank of $N$ is 3 and $N$ is a $3\times 3$ matrix,
					\[
						\Rref(N) = \mat{1&0&0\\0&1&0\\0&0&1},
					\]
					and so the column space of $\Rref(N)$ is the span of the vectors
					$\mat{1\\0\\0},\mat{0\\1\\0},\mat{0\\0\\1}$ and so is all of $\R^3$.  Thus, 
					the column spaces of $N$ and $\Rref(N)$ must be the same.

					The assumption that $\Rank(N)=3$ cannot be dropped.  If $\Rank(N)<3$, a counter-example
					like that in part (a) could be constructed.
				\end{quote}
		\end{enumerate}
		
		\item For a linear transformation $L:\R^3\to\R^3$, we have the following information:
			\[
				L\mat{1\\1\\0}=\mat{2\\1\\1}\qquad
				L\mat{-1\\1\\0}=\mat{-3\\3\\3}\qquad
				L\mat{0\\0\\1}=\mat{0\\0\\0}.
			\]
		\begin{enumerate}
			\item Write down a matrix for $L$.
				\begin{quote}
					$L:\R^3\to\R^3$ is a linear transformation, which by definition means 
					$L(\vec u+\vec v)=L\vec u+L\vec v$ and $L(k\vec u)=kL\vec u$.  Since $L$ is
					linear, we know that we can write down a matrix for $L$.

					The easiest way to write down a matrix for $L$ is to computer
					$L\vec e_1,L\vec e_2$, and $L\vec e_3$ which will give the first,
					second, and third columns of a matrix for $L$.  We will use linearity
					to our advantage.  

					First, notice that 
					\[
						\vec e_1=\tfrac{1}{2}\mat{1\\1\\0}-\tfrac{1}{2}\mat{-1\\1\\0}\qquad
						\vec e_2=\tfrac{1}{2}\mat{1\\1\\0}+\tfrac{1}{2}\mat{-1\\1\\0}\qquad
						\vec e_3=\mat{0\\0\\1}.
					\]
					Using linearity, we now compute
					\[
						L\vec e_1 = L\left(\tfrac{1}{2}\mat{1\\1\\0}-\tfrac{1}{2}\mat{-1\\1\\0}\right)
						=\tfrac{1}{2}L\mat{1\\1\\0}-\tfrac{1}{2}L\mat{-1\\1\\0}
						=\tfrac{1}{2}\mat{2\\1\\1}-\tfrac{1}{2}\mat{-3\\3\\3}= \mat{\tfrac{5}{2}\\-1\\-1}
					\]
					\[
						L\vec e_2 = L\left(\tfrac{1}{2}\mat{1\\1\\0}+\tfrac{1}{2}\mat{-1\\1\\0}\right)
						=\tfrac{1}{2}L\mat{1\\1\\0}+\tfrac{1}{2}L\mat{-1\\1\\0}
						=\tfrac{1}{2}\mat{2\\1\\1}+\tfrac{1}{2}\mat{-3\\3\\3}= \mat{-\tfrac{1}{2}\\2\\2}
					\]
					\[
						L\vec e_3=\mat{0\\0\\0}.
					\]
					This means a matrix for $L$ is
					\[
						\mat{\tfrac{5}{2}&-\tfrac{1}{2}&0\\-1&2&0\\-1&2&0}.
					\]
				\end{quote}
			\item Describe the range of $L$ as a point, line, plane, or hyperplane and
				give a basis for the range of $L$.
				\begin{quote}
					From part (a), we can see that the rank of $L$ is 2. Therefore, the range
					of $L$ will be a two dimensional subspace, making it a plane.
				\end{quote}
			\item Describe the null space of $L$ as a point, line, plane, or hyperplane and
				give a basis for the null space of $L$.
				\begin{quote}
					The \emph{rank-nullity} theorem states that $\Rank(L)+\Nullity(L)=\#$ of columns in a
					matrix representation of $L$, or equivalently, the dimension of the target space. 
					In this case, the target space is
					$\R^3$.  Thus $\Rank(L)+\Nullity(L)=3$.  Since $\Rank(L)=2$, we know
					$\Nullity(L)=1$ and so the dimension of the null space of $L$ is 1, making it a line.
				\end{quote}
		\end{enumerate}

	\end{enumerate}
\end{document}
